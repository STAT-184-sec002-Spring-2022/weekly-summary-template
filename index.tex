% Options for packages loaded elsewhere
\PassOptionsToPackage{unicode}{hyperref}
\PassOptionsToPackage{hyphens}{url}
\PassOptionsToPackage{dvipsnames,svgnames,x11names}{xcolor}
%
\documentclass[
  letterpaper,
  DIV=11,
  numbers=noendperiod]{scrartcl}

\usepackage{amsmath,amssymb}
\usepackage{lmodern}
\usepackage{iftex}
\ifPDFTeX
  \usepackage[T1]{fontenc}
  \usepackage[utf8]{inputenc}
  \usepackage{textcomp} % provide euro and other symbols
\else % if luatex or xetex
  \usepackage{unicode-math}
  \defaultfontfeatures{Scale=MatchLowercase}
  \defaultfontfeatures[\rmfamily]{Ligatures=TeX,Scale=1}
\fi
% Use upquote if available, for straight quotes in verbatim environments
\IfFileExists{upquote.sty}{\usepackage{upquote}}{}
\IfFileExists{microtype.sty}{% use microtype if available
  \usepackage[]{microtype}
  \UseMicrotypeSet[protrusion]{basicmath} % disable protrusion for tt fonts
}{}
\makeatletter
\@ifundefined{KOMAClassName}{% if non-KOMA class
  \IfFileExists{parskip.sty}{%
    \usepackage{parskip}
  }{% else
    \setlength{\parindent}{0pt}
    \setlength{\parskip}{6pt plus 2pt minus 1pt}}
}{% if KOMA class
  \KOMAoptions{parskip=half}}
\makeatother
\usepackage{xcolor}
\setlength{\emergencystretch}{3em} % prevent overfull lines
\setcounter{secnumdepth}{-\maxdimen} % remove section numbering
% Make \paragraph and \subparagraph free-standing
\ifx\paragraph\undefined\else
  \let\oldparagraph\paragraph
  \renewcommand{\paragraph}[1]{\oldparagraph{#1}\mbox{}}
\fi
\ifx\subparagraph\undefined\else
  \let\oldsubparagraph\subparagraph
  \renewcommand{\subparagraph}[1]{\oldsubparagraph{#1}\mbox{}}
\fi


\providecommand{\tightlist}{%
  \setlength{\itemsep}{0pt}\setlength{\parskip}{0pt}}\usepackage{longtable,booktabs,array}
\usepackage{calc} % for calculating minipage widths
% Correct order of tables after \paragraph or \subparagraph
\usepackage{etoolbox}
\makeatletter
\patchcmd\longtable{\par}{\if@noskipsec\mbox{}\fi\par}{}{}
\makeatother
% Allow footnotes in longtable head/foot
\IfFileExists{footnotehyper.sty}{\usepackage{footnotehyper}}{\usepackage{footnote}}
\makesavenoteenv{longtable}
\usepackage{graphicx}
\makeatletter
\def\maxwidth{\ifdim\Gin@nat@width>\linewidth\linewidth\else\Gin@nat@width\fi}
\def\maxheight{\ifdim\Gin@nat@height>\textheight\textheight\else\Gin@nat@height\fi}
\makeatother
% Scale images if necessary, so that they will not overflow the page
% margins by default, and it is still possible to overwrite the defaults
% using explicit options in \includegraphics[width, height, ...]{}
\setkeys{Gin}{width=\maxwidth,height=\maxheight,keepaspectratio}
% Set default figure placement to htbp
\makeatletter
\def\fps@figure{htbp}
\makeatother

\KOMAoption{captions}{tableheading}
\makeatletter
\@ifpackageloaded{tcolorbox}{}{\usepackage[many]{tcolorbox}}
\@ifpackageloaded{fontawesome5}{}{\usepackage{fontawesome5}}
\definecolor{quarto-callout-color}{HTML}{909090}
\definecolor{quarto-callout-note-color}{HTML}{0758E5}
\definecolor{quarto-callout-important-color}{HTML}{CC1914}
\definecolor{quarto-callout-warning-color}{HTML}{EB9113}
\definecolor{quarto-callout-tip-color}{HTML}{00A047}
\definecolor{quarto-callout-caution-color}{HTML}{FC5300}
\definecolor{quarto-callout-color-frame}{HTML}{acacac}
\definecolor{quarto-callout-note-color-frame}{HTML}{4582ec}
\definecolor{quarto-callout-important-color-frame}{HTML}{d9534f}
\definecolor{quarto-callout-warning-color-frame}{HTML}{f0ad4e}
\definecolor{quarto-callout-tip-color-frame}{HTML}{02b875}
\definecolor{quarto-callout-caution-color-frame}{HTML}{fd7e14}
\makeatother
\makeatletter
\makeatother
\makeatletter
\makeatother
\makeatletter
\@ifpackageloaded{caption}{}{\usepackage{caption}}
\AtBeginDocument{%
\ifdefined\contentsname
  \renewcommand*\contentsname{Table of contents}
\else
  \newcommand\contentsname{Table of contents}
\fi
\ifdefined\listfigurename
  \renewcommand*\listfigurename{List of Figures}
\else
  \newcommand\listfigurename{List of Figures}
\fi
\ifdefined\listtablename
  \renewcommand*\listtablename{List of Tables}
\else
  \newcommand\listtablename{List of Tables}
\fi
\ifdefined\figurename
  \renewcommand*\figurename{Figure}
\else
  \newcommand\figurename{Figure}
\fi
\ifdefined\tablename
  \renewcommand*\tablename{Table}
\else
  \newcommand\tablename{Table}
\fi
}
\@ifpackageloaded{float}{}{\usepackage{float}}
\floatstyle{ruled}
\@ifundefined{c@chapter}{\newfloat{codelisting}{h}{lop}}{\newfloat{codelisting}{h}{lop}[chapter]}
\floatname{codelisting}{Listing}
\newcommand*\listoflistings{\listof{codelisting}{List of Listings}}
\makeatother
\makeatletter
\@ifpackageloaded{caption}{}{\usepackage{caption}}
\@ifpackageloaded{subcaption}{}{\usepackage{subcaption}}
\makeatother
\makeatletter
\@ifpackageloaded{tcolorbox}{}{\usepackage[many]{tcolorbox}}
\makeatother
\makeatletter
\@ifundefined{shadecolor}{\definecolor{shadecolor}{rgb}{.97, .97, .97}}
\makeatother
\makeatletter
\makeatother
\ifLuaTeX
  \usepackage{selnolig}  % disable illegal ligatures
\fi
\IfFileExists{bookmark.sty}{\usepackage{bookmark}}{\usepackage{hyperref}}
\IfFileExists{xurl.sty}{\usepackage{xurl}}{} % add URL line breaks if available
\urlstyle{same} % disable monospaced font for URLs
\hypersetup{
  pdftitle={Weekly Summary Template},
  pdfauthor={Miranda Goodman},
  colorlinks=true,
  linkcolor={blue},
  filecolor={Maroon},
  citecolor={Blue},
  urlcolor={Blue},
  pdfcreator={LaTeX via pandoc}}

\title{Weekly Summary Template}
\author{Miranda Goodman}
\date{}

\begin{document}
\maketitle
\ifdefined\Shaded\renewenvironment{Shaded}{\begin{tcolorbox}[borderline west={3pt}{0pt}{shadecolor}, sharp corners, boxrule=0pt, breakable, interior hidden, frame hidden, enhanced]}{\end{tcolorbox}}\fi

\renewcommand*\contentsname{Table of contents}
{
\hypersetup{linkcolor=}
\setcounter{tocdepth}{3}
\tableofcontents
}
\begin{center}\rule{0.5\linewidth}{0.5pt}\end{center}

\hypertarget{tuesday-feb-7}{%
\subsection{Tuesday, Feb 7}\label{tuesday-feb-7}}

\begin{tcolorbox}[enhanced jigsaw, arc=.35mm, toptitle=1mm, breakable, colback=white, bottomrule=.15mm, opacityback=0, colbacktitle=quarto-callout-important-color!10!white, title=\textcolor{quarto-callout-important-color}{\faExclamation}\hspace{0.5em}{TIL}, bottomtitle=1mm, titlerule=0mm, leftrule=.75mm, coltitle=black, rightrule=.15mm, left=2mm, toprule=.15mm, colframe=quarto-callout-important-color-frame, opacitybacktitle=0.6]

Include a \emph{very brief} summary of what you learnt in this class
here.

Today, I learnt the following concepts in class:

\begin{enumerate}
\def\labelenumi{\arabic{enumi}.}
\tightlist
\item
  Interpretation of regression coeficients
\item
  Categorical Covariates
\item
  How to change the baselines for a model
\end{enumerate}

\end{tcolorbox}

Provide more concrete details here. You can also use
footenotes\footnote{You can include some footnotes here} if you like

In class we learned about how to interpret regression coeficients. This
meant how to interpret \(\beta_0\) and \(\beta_1\).

Given this regression model: \[
y_i = \beta_0 + \beta_1 x_i + \epsilon_i
\] \(y_i\) is the response variable \(x_i\) is the covariate
\(\epsilon_i\) is the error \(\beta_0\) and \(\beta_1\) are the
regression coefficients \(\beta_0\) is the intercept and \(\beta_1\) is
the slope

Example: For some covariate \(x_0\) the expected value for \(y(x_0)\) is
given by the equation \(y(x_0) = \beta_0 + \beta_1 x_0\)

The expected value of \(x_0 + 1\) is given by:

\[
\begin{align}
y(x_0 +1) &= \beta_0 + \beta_1 \times (x_0 + 1)\\
&= \beta_0 + \beta_1 x_0 + \beta_1\\
\end{align}
\] In class we learned about categorical covariates and regression
models:

In classs we used the example of the iris data set.

\begin{verbatim}
iris %>% head() %>% kable()
\end{verbatim}

We looked at if there is a relationship between `species' and
`sepal.length'.

EDA:

\begin{verbatim}
y <- iris$Sepal.Length
x <- iris$Species

boxplot(Sepal.Length ~ SPecies, df)
\end{verbatim}

Linear Regression Model:

\begin{verbatim}
reg_model <- lm(Sepal.Length ~ Species, iris)
reg_model
\end{verbatim}

Regression model:

\[
y_i = \beta_0 + \beta_1 x_i
\]

where \(x_i \in \{ \setosa, \versicolor, \virginica)\)\} We get the
following models:

\$y\_i = \beta\_0 + \beta\_1 x\_i = \$ `setosa' \$y\_i = \beta\_0 +
\beta\_1 x\_i = \$ `versicolor' \$y\_i = \beta\_0 + \beta\_1 x\_i = \$
`virginica'

Interpretation:

\#\#\#Intercept

\(\beta_0\) is the expected \(y\) value when \(x\) belongs to the base
category.

\#\#\#\#Slopes \(\beta_1\) with the name `Species.versicolor' represents
the following

`(Intercept)' = \(y(x = \texttt{setosa})\)

'Species.versicolor =
\(y(x = \texttt{versicolor}) - y(x = \texttt{setosa})\)
'Species.verginica =
\(y(x = \texttt{verginica}) - y(x = \texttt{setosa})\)

In class we learned how to change the baselines for a model: To change
the baseline to virginica:

\begin{verbatim}
iris$Species
iris$Species <- relevel(iris$Species, "virginica")
iris$SPecies
\end{verbatim}

Now we can run the regression model:

\begin{verbatim}
new_reg_model <- lm(Sepal.Length ~ Species, iris)
new_reg_model
\end{verbatim}

\begin{verbatim}
#| output: false
library(dplyr)
library(purrr)
\end{verbatim}

For example: in class we learnt we learnt about the \texttt{map}
function from the \texttt{purrr} package.

\texttt{results=\textquotesingle{}hide\textquotesingle{},\ fig.width=7,\ fig.height=7\}\ par(mfrow=c(3,\ 3),\ mar=c(3.5,\ 3.5,\ 2,\ 1),\ mgp=c(2.4,\ 0.8,\ 0))\ map(1:9,\ function(i)rnorm(1000)\ \%\textgreater{}\%\ hist(.,\ main=i,\ col=i))}

\hypertarget{thursday-feb-9}{%
\subsection{Thursday, Feb 9}\label{thursday-feb-9}}

\begin{tcolorbox}[enhanced jigsaw, arc=.35mm, toptitle=1mm, breakable, colback=white, bottomrule=.15mm, opacityback=0, colbacktitle=quarto-callout-important-color!10!white, title=\textcolor{quarto-callout-important-color}{\faExclamation}\hspace{0.5em}{TIL}, bottomtitle=1mm, titlerule=0mm, leftrule=.75mm, coltitle=black, rightrule=.15mm, left=2mm, toprule=.15mm, colframe=quarto-callout-important-color-frame, opacitybacktitle=0.6]

Include a \emph{very brief} summary of what you learnt in this class
here.

Today, I learnt the following concepts in class:

\begin{enumerate}
\def\labelenumi{\arabic{enumi}.}
\tightlist
\item
  Multiple Regression
\item
  How to apply multiple regression to a data set
\item
  How to interpret the coefficients
\end{enumerate}

\end{tcolorbox}

Provide more concrete details here, e.g.,

In class we learned about multiple regression. \[
y - \beta_0 + \beta_1 x_1 + \beta_2 x_2 + \dots \beta_p x_p + \epsilon
\] where in the data \(y_1, y_2 \dots y_n\) is the response variable and
\(x_1, x_2, \dots x_n\) is the covariates.

The model:

\[
y_i = \beta_0 + \beta_1 x_{1,i} + \beta_2 x_{2,i} + \dots + \beta_p x_{p, i} + \epsilon_i
\]

Applying multiple regression to a credit data set:

\begin{verbatim}
library(ISLR2)
attatch(credit)

df <- Credit %>% tibble()
colnames(df) <- tolower(colnames(df))
df
\end{verbatim}

Looking at income, rating, limit

\begin{verbatim}
df3 <- df %>% select(c("income", "rating", "limit"))
df3
\end{verbatim}

\begin{verbatim}
fig <- plot_ly(df3, x=~income, y =~rating, z=~limit)
fig %>% add_markers()
\end{verbatim}

\begin{verbatim}
model <- lm(limit ~ income + rating, df3)
model
\end{verbatim}

\begin{verbatim}
ranges <- df3 %>%
  select(income, rating) %>%
  colnames() %>%
  map(\(x) seq(0.1 * min(df3[x]), 1.1 * max(df3[x]), length.out = 50))

b <- model$coefficients
z <- outer(
  ranges[[1]],
  ranges[[2]],
  Vectorize(function(x2, x3) {
    b[1] + b[2] * x2 + b[3] * x[3]
  })
)

fig %>%
  add_surface(x = ranges[[1]], y = ranges[[2]], z = t(z),
              alpha=.3) %>%
  add_markers()
\end{verbatim}

We also learned how to interpret the coefficients:

\(\beta_0\) is the expected value of \(y\) when \(income = 0\) and
\(rating = 0\) \(beta_1\) is saying that if \(rating\) is held constant
and \(income\) changed by 1 unit, then the corresponding change in the
`limit' is \(0.5573\) \(\beta_2\) is saying that if `income' is held
constant and `rating' changes by \(1\) unit, then the corresponding
change in `limit' is \(14.771\)

In class we learnt how to use the \texttt{map} function to create
multiple regression diagnostic plots

\begin{verbatim}
par(mfcol=c(2, 3), mar=c(3.5, 3.5, 2, 1), mgp=c(2.4, 0.8, 0))
mtcars %>%
  split(.$cyl) %>%
  map(~ lm(mpg ~ wt, data = .x)) %>%
  map(function(x)plot(x, which=c(1, 2)))
\end{verbatim}



\end{document}
